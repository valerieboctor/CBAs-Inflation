 \documentclass[12pt]{article}
\usepackage[capposition=top]{floatrow}
\usepackage{graphicx}
\usepackage{amssymb}
\usepackage{amsmath}
\usepackage{amsfonts}
\usepackage{float}
\usepackage{eurosym}
\usepackage{csvsimple}
\usepackage{booktabs}
\usepackage{tikz}
\usepackage{hyperref}
\usepackage{ulem}
\usepackage{caption}
\usepackage{tablefootnote}
\usepackage{ntheorem}
\usepackage{sectsty}
\usepackage[justification=centering,textfont={sc},labelfont={rm}]{caption}
\usepackage{geometry}
\geometry{margin=.25in}
\usepackage[numbers, square]{natbib}
\usepackage{graphicx}
\usepackage{setspace}
\onehalfspacing
\begin{document}
	\title{\small{Bringing Expectations to the Collective Bargaining Table: Evidence from Brazilian Firms}}
	\author{\small{Valerie R. Boctor}\thanks{This project is a collaboration with Roberto Hsu Rocha. This document was written by Valerie Boctor. All mistakes are my own.}}
	\date{\vspace{-1.5cm}}
	\maketitle
	\abstract{We use matched-employer and employee data from Relacao Anual de Informacoes Socials (RAIS) as well as information on collective bargaining outcomes from Sistemas Mediador to study the effects of a major inflationary news shock on collective bargaining outcomes and macroeconomic implications in Brazil. A major news shock occurred during the impeachment of former President Dilma Roussef in early 2016: Inflation expectations plummeted and stabilized from 11.5\% in February 2016 to around 3\% by May 2017. Using variation in the timing of collective bargaining agreements at the firm level, we assess the impact of inflation expectations on firms' wages and employment. Furthermore, we examine the extent to which downward nominal wage rigidity binds at the firm level, as well as alternative avenues firms may use to compensate for suboptimal real wages. We will use a we will use an adapted staggered pricing model assess the macroeconomic implications of collective bargaining activity driven by inflationary news shocks. 
	\begin{center}\textbf{Data} \end{center}
	In addition to publicly available aggregate inflation expectations from the Brazilian Central Bank, we use two main data sets to carry out this study: RAIS, which contains highly detailed information about individual workers, employers and unions, as well as the collective bargaining data from Sistemas Mediador. Using the latter database, we are able to obtain detailed information about firm-to-worker collective bargaining agreements, including bargained wages, effective dates, and firm IDs. While this data is arguably sufficient to answer the main research questions, we are also in the process of obtaining firm-level expectations of their own prices and costs. This data would be useful in the way of establishing an empirical relationship between firms' price expectations and resultant wages. This information may also help us to assess whether firms are behaving in a manner consistent with a staggered pricing scheme. 
	
	\begin{center} \textbf{Economic Motivation} \end{center}
	This project is motivated by the sheer size of the drop in inflation expectations around the Roussef's impeachment and anticipation of a major regime shift. This event can be viewed as a ``modern-day Volcker shock,'' providing ample variation to study the effects on collective bargaining activity using unique, highly detailed micro data. This project also has implications for our understanding of forward guidance, which is typically thought to operate through consumption and investment. Our project adds to these by establishing empirical evidence that the collective bargaining channel may also play an important role in macroeconomic performance. Our research also provides an empirical test for Tenreyro \& Olivei's (2007 AER) conjecture that the timing of monetary policy shocks in the US matters due to the seasonality in wage rigidity, which is ostensibly driven by the timing of wage contracts. 
		\begin{center} \textbf{Committee Members} \end{center} Yuriy Gorodnichenko (Dissertation Advisor), Emi Nakamura (tentative Committee Chair), Benjamin Schoefer, David Card
\end{document}