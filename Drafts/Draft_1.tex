 \documentclass[12pt]{article}
\usepackage[capposition=top]{floatrow}
\usepackage{graphicx}
\usepackage{amssymb}
\usepackage{amsmath}
\usepackage{amsfonts}
\usepackage{float}
\usepackage{eurosym}
\usepackage{csvsimple}
\usepackage{booktabs}
\usepackage{tikz}
\usepackage{hyperref}
\usepackage{ulem}
\usepackage{caption}
\usepackage{tablefootnote}
\usepackage{ntheorem}
\usepackage{sectsty}
\usepackage[justification=centering,textfont={sc},labelfont={rm}]{caption}
\usepackage{geometry}
\geometry{margin=.5in}
\usepackage[numbers, square]{natbib}
\usepackage{graphicx}
\usepackage{setspace}
\onehalfspacing
\begin{document}
	\title{Bringing Expectations to the Collective Bargaining Table: Evidence from Brazilian Firms}
	\author{Valerie R. Boctor \\ Roberto Hsu Rocha}
	\date{\today}
	\maketitle
	\abstract{\textit{Ideal: We use matched employer-employee data and firm-level inflation expectations to examine the role of inflationary news shocks on collective bargaining outcomes in Brazil. Our quasi-natural experiment comes from firm-level variation in the timing of collective bargaining agreements (CBAs) around inflationary news shock. That is, firms with CBAs just before an inflationary news shock may have wage and employment levels that differ systematically from firms whose CBAs fall just after the shock. For firms that settled contracts just before the shock, real wages were set too high, leading to reductions in employment and investment at the firm level. Furthermore, we adapt the Diamond-Mortensen-Pissarides model to include inflation expectations to demonstrate that the macroeconomic effects of collective bargaining channel of inflationary news shocks.}}


	\section{Introduction}
	\subsection{Literature Review}
\end{document}